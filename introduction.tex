\chapter{Introduction}

Advances in manufacturing and hardware design have made mobile robots more accessible for not only research and industrial purposes, but to the general public. However, autonomous robots have largely been limited to indoor, carefully controlled environments. Before autonomous robots can be more widely deployed, they must be able to accurately observe and represent unstructured, dynamic environments. Dense sensors such as light field cameras are becoming cheaper and lighter, promising to allow autonomous robots to acquire detailed measurements of these complex surroundings. To fully utilise the potential of these advancing technologies, improvements in observer design are required to generate more accurate and detailed descriptions of these environments from dense measurements.

One method of estimating the state of a complex environment is with an \textit{infinite-dimensional observer}. Typically, observers for infinite-dimensional systems are extensions of finite-dimensional Luenberger observers. Unfortunately, this design approach is only able to guarantee convergence for linear systems. Developing a theory of symmetry-preserving, infinite-dimensional observers would simplify the design process for nonlinear systems and result in observers with improved convergence properties.

This research project serves as an initial exploration into the design of symmetry-preserving, infinite-dimensional observers. An infinite-dimensional system will be simplified and an observer will be designed to estimate a finite-dimensional state. The potential for a symmetry-preserving, infinite-dimensional observer to improve performance will be explored. 

This report presents the novel implementation of an observer to estimate the state of a rigid cube from range measurements. Section \ref{sec:literature} reviews the current state of observer design methods for infinite-dimensional systems. Recent work in the development of design methodologies for symmetry preserving observers is described. Particular attention is paid to a dense optical flow estimator that will be particularly relevant to this research.
 
Chapter \ref{chap:background} provides theory on Lie groups, rigid body state representation and state observer concepts that will be applied in the observer design.

The cube state estimation problem that is the focus of this research is described in detail in Chapter \ref{chap:problem}. The place of this problem within the larger area of symmetry-preserving, infinite-dimensional observer research is defined.  

Chapter \ref{chap:simulation} provides a detailed description of the simulation implementation, including the observer update function design. The performance of the observer in estimating the state of stationary and moving cubes is assessed. It is shown that almost global convergence is achieved for orientation and size correction for stationary cubes, though the position update only converges in special cases.

Chapter \ref{chap:experiment} details steps taken to experimentally validate the results of the simulation. Range measurement data is collected and used to develop an error distribution model for the Hokuyo UBG-04LX-F01 sensor. Range measurements of a cube of known trajectory are taken for future performance testing.

\section{Literature Review}
\textbf{should this be a chapter???}

Use dense sensors to estimate infinite dimensional state of environment can be achieved through an infinite dimensional observer. Richer theory can be gained by taking advantage of symmetries of the system. Will review progress in infinite dimensional, symmetry preserving observers.

\subsection{infinite-dimensional observers}
In many real world systems the dependent variables are functions of one or more spatial variables. These spatial variables form a continuum of ??? which means an infinite number of parameters are required to describe the state. Such systems are termed infinite dimensional systems, or distributed parameter systems. Their dynamics are modelled by a partial differential equation (PDE). When a state estimate is required but direct measurement of the state with sensors is difficult or impossible, a state observer is employed. A state observer estimates the state of a system using the difference between measured and predicted outputs of the system.

\textbf{linear:}\\
Observer theory for infinite dimensional \textit{linear} systems has been widely studied. The techniques used are typically extensions of Luenberger observers and Kalman filter methods used to observe finite dimensional systems.

A simplified approach is to use a spatial discretisation method such as finite difference or finite element to reduce the infinite dimensional system to a finite dimensional one. From here, finite dimensional observer design techniques can be used. This is known as the early lumping method, and was employed by Stavroulakis \cite{stavroulakis1973design} who implemented a finite dimensional observer as part of a control system for an infinite dimensional systems.

The early lumping approach suffers from \textit{spillover}, a phenomenon where performance is affected by the neglected dynamics \cite{meirovitch1983problem}. Harkort \cite{harkort2011finite} recently developed an observer based control scheme that reduced this effect by using modelled outputs rather than true measurements to reduce the effect of the neglected dynamics.

More accurate observers can be designed with the late lumping approach which uses the infinite dimensional model of the system in the observer design itself. The result is what is an actual infinite dimensional observer that is discretised later for practical implementation. These methods are typically extensions of Kalman or Luenberger observers. 
Gressang \cite{gressang1975observers} extended the Luenberger observer to infinite dimensional systems whose state space was an abstract Banach space - dynamics defined by infinitesimal generator - sort of like basis ? of semigroup.

Smyshlyaev \cite{smyshlyaev2005backstepping} developed an exponentially converging backstepping observer for systems governed by parabolic PDEs.

Ramdani introduced forward and backward observers \cite{ramdani2010recovering}.

Haine \cite{haine2014recovering} applied Ramdani's method, studied convergence properties.

\textbf{nonlinear}

-no universal approach for observer design for infinite-dimensional nonlinear systems\\
-some methods for special case - infinite dimensional bilinear systems. \cite{xu1995observer,bounit1997observers}
-for finite-dimensional nonlinear systems, common design methods are: linearisation (ie EKF), lyapunov method, sliding mode, high gain

\subsection{symmetry preserving observers}
\subsubsection{early work}
\subsubsection{bonnabel et al}
\subsubsection{trumpf, mahony et al}
\subsubsection{juan's work - in detail}

