\chapter*{Conclusion}

An observer has been designed to estimate the state of a rigid cube from sparse range measurements. It has been shown that by appropriately defined the sensor trajectory, the sparse range measurements can be used to approximate those of a dense sensor.

The observer is globally convergent when correcting the orientation and size of stationary cubes for trajectories where two or three cube faces are visible. This implementation also reveals the limitations of an observer update function that does not consider symmetry. The position update only works in the special case that a single cube face is visible. In order to simultaneously correct position and orientation, the update function must be invariant to actions of $\mathbf{SE}(3)$.
It is recommended that the position of the cube be updated by computing the centre of mass of a history of predicted and measured points, and defining the position update as the difference of these. This history of points would be added as an augmented state variable. A similar approach could be used to correct the orientation and size of the cube state estimate. Such an approach would be far more resistant to noise.

It is recommended that the position and orientation updates are combined into a function that acts on either the screw, twist or wrench as a whole, rather than extracting the rotation matrix or position vector. Such an update function should be invariant to actions of $\mathbf{SE}(3)$, leading to improved convergence properties over a wider region of trajectories.

A major simplification of this observer design was the representation of the infinite-dimensional environment as a single target object and background object. Future work should attempt to model a more complex environment that is a better representation of an infinite-dimensional system. The triangular mesh method of modelling rigid bodies in the simulation implementation would allow for complex, deformable surfaces to be represented with few changes required to the code. Rather than separating the range measurements corresponding to the cube and background, all measurements should be used. The observer update function could then be driven by the difference in measured and predicted ranges, rather than the separate values. The benefit of this addition is that the observer would have a form more like the traditional Luenberger observer that is used in existing symmetry-preserving observer design methodologies \cite{bonnabel2009non,mahony2013observers}.


