\chapter{Simulation}
\section{Implementation}
\subsection{Sensor modelling}
\textbf{motion:}\\
Frame \{B\} aligned with moving rigid body, + inertial frame \{A\}.\\
screw matrix: ${^{A}_{A}\mathbf{S}^{}_{B}}(t+\delta t) = {^{A}_{A}\mathbf{S}^{}_{B}(t)}e^{\delta t {^{B}_{A}\mathbf{T}^{}_{B}(t)}}$
			
twist matrix: ${^{B}_{A}\mathbf{T}^{}_{B}}(t+\delta t) = {^{B}_{A}\mathbf{T}^{}_{B}(t)} + \delta t {^{B}_{A}\mathbf{W}^{}_{B}(t)}$

wrench matrix: ${^{B}_{A}\mathbf{W}^{}_{B}}(t+\delta t) = {^{B}_{A}\mathbf{W}^{}_{B}(t)}$

\textbf{scanning:}
sensor parameters:
\begin{itemize}
\item field of view
\item angular resolution
\item time per scan
\item measurement resolution
\end{itemize}
From these parameters, create vector $\mathbf{n}(t)$. At each time, n is either a unit vector in direction of scan, in body fixed frame, or returns no value for when sensor is not returning a measurement (outside FOV)

scan direction: ${^{B}_{B}\mathbf{n}^{}_{?}}(t)$

\subsection{Environment Modelling}
Environment composed of one or more rectangular prisms. These objects are modelled as an ordered set of 8 points in the inertial reference frame, and 12 triangles. Each triangle is a set of 3 integers, indicating the index of the three points that make up its vertexes.
The position of each of these points at each time is determined using the screw matrix and side lengths of the object.

DIAGRAM HERE

\subsection{Measurement Modelling}
	\subsubsection{Range Computation}
	Given the screw matrix (in the inertial frame) and scan direction (in the body fixed frame) of the sensor, the position and scan direction in the inertial frame are determined.
	The distance to the nearest environment object from this point, along the scan direction is determined with a triangle-ray intersection formula.
	
	inputs:
	\begin{itemize}
	\item sensor position: ${^{A}_{A}\mathbf{p}^{}_{B}}(t)$
	\item scan direction: ${^{A}_{A}\mathbf{n}^{}_{?}}(t)$
	\item object points: ${^{A}_{A}\mathbf{E}^{}_{?}}(t)$
	\item triangle indexes:	$Tri$
	\end{itemize}
	
	how to describe computation? pseudocode?\\
	
	
	
	\subsubsection{Noise Modelling}
	Different noise models for different sensors
	\begin{equation}
		\hat{r}(t) = f_s(r(t),\theta(t),\phi(k))
	\end{equation}
	where $\theta(t)$ is incidence angle of measurement, $\phi$ is surface properties of object $k$ that was measured, $f$ is some function for noise model of particular sensor.
	
\subsection{Observer implementation}
\section{Results}