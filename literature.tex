\section{Literature Review}
\textbf{should this be a chapter???}

Use dense sensors to estimate infinite dimensional state of environment can be achieved through an infinite dimensional observer. Richer theory can be gained by taking advantage of symmetries of the system. Will review progress in infinite dimensional, symmetry preserving observers.

\subsection{infinite-dimensional observers}
In many real world systems the dependent variables are functions of one or more spatial variables. These spatial variables form a continuum of ??? which means an infinite number of parameters are required to describe the state. Such systems are termed infinite dimensional systems, or distributed parameter systems. Their dynamics are modelled by a partial differential equation (PDE). When a state estimate is required but direct measurement of the state with sensors is difficult or impossible, a state observer is employed. A state observer is a filter that provides an estimate of the state of a system using the difference between measured and predicted outputs of the system.

\subsection{linear}
Observer theory for infinite dimensional \textit{linear} systems has been widely studied. The techniques used are typically extensions of Luenberger observers and Kalman filter methods used to observe finite dimensional systems.

A simplified approach is to use a spatial discretisation method such as finite difference or finite element to reduce the infinite dimensional system to a finite dimensional one. From here, finite dimensional observer design techniques can be used. This is known as the early lumping method, and was employed by Stavroulakis \cite{stavroulakis1973design} who implemented a finite dimensional observer as part of a control system for an infinite dimensional systems.

The early lumping approach suffers from \textit{spillover}, a phenomenon where performance is affected by the neglected dynamics \cite{meirovitch1983problem}. Harkort \cite{harkort2011finite} recently developed an observer based control scheme that reduced this effect by using modelled outputs rather than true measurements to reduce the effect of the neglected dynamics.

More accurate observers can be designed with the late lumping approach which uses the infinite dimensional model of the system in the observer design itself. The result is what is an actual infinite dimensional observer that is discretised later for practical implementation. These methods are typically extensions of Kalman or Luenberger observers. 
Gressang \cite{gressang1975observers} extended the Luenberger observer to infinite dimensional systems whose state space was an abstract Banach space - dynamics defined by infinitesimal generator - sort of like basis ? of semigroup.

Smyshlyaev \cite{smyshlyaev2005backstepping} developed an exponentially converging backstepping observer for systems governed by parabolic PDEs.

Ramdani introduced forward and backward observers \cite{ramdani2010recovering}.

Haine \cite{haine2014recovering} applied Ramdani's method, studied convergence properties.

\subsection{nonlinear}
There is no universal approach for observer design for infinite-dimensional nonlinear systems\\
One approach is to linearise the system, then apply a method for infinite dimensional linear observer design. Common linearisation methods are Lyapunov methods, extended linearisation and the Lie-algebraic approach \cite{primbs1996survey}.
For the special case of bilinear systems, Xu \cite{xu1995observer} designed an infinite dimensional observer that converged given certain inputs. some methods for special case - infinite dimensional bilinear systems. Bounit \cite{bounit1997observers} designed Kalman and Luenberger type observers for infinite dimensional bilinear systems. 

Trend - many design methods for infinite dimensional nonlinear systems rely on linearisation techniques. Differentiable function can be approximated by first order taylor expansion around a point. Linear approximation of system around an equilibrium point. Theory of Kalman filters, Luenberger observers can be applied. This simplification relies on dynamics of system at the point -  behaviour not the same throughout. Result is that observer only converges if the initial state estimate is within a local neighbourhood of the true state. Global converge is not guaranteed. 
To achieve this, direction into geometry, symmetry of system. Particularly powerful when dealing with symmetries is theory of Lie groups.
Investigation into observers for systems on Lie groups:

\subsection{symmetry preserving observers}
The idea behind symmetry preserving observers is to take advantage of invariances of the system. Design an observer around an equilibrium point in such a way that it can be extended to a wider set of points.

\subsubsection{early work}
Observer design taking geometry into account not a new idea. Early investigation by Marcus \cite{marcus1984algebraic} into algebraic and geometric methods for nonlinear filter design showed idea had promise.
An important work by Salcudean - eventually exponential, globally converging observer - rigid bodies, attitude.
Work that has led to research today is by Aghannon \& Rouchon \cite{aghannan2002invariant} - invariant observer construction method.
Led to work by Maithripala \cite{maithripala2005intrinsic}. 

Now, two active research groups.
\subsubsection{bonnabel et al}
Design observer around equilibrium point - extend to nonlinear - behaves well around continuum of equilibrium points.
invariant observer - luenberger observer where invariant frame used to construct invariant output error.
\subsubsection{trumpf, mahony et al}
lifted system

\subsection{infinite dimensional symmetry preserving observers}
Very little published,\\
Auroux \& Bonnabel \cite{auroux2011symmetry}, Zarrouati \cite{zarrouati2013augmented} \\
juan's work \cite{adarvefiltering} - in detail
