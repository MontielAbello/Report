\section{Literature Review}
\textbf{should this be a chapter???}

The use of dense sensors allows for a more accurate estimation of the state of an infinite-dimensional system such as a complex, real-world environment. The theory of infinite-dimensional observers is required to fully utilise this information. This section will review the current state of design methodologies and implementations for infinite-dimensional observers. Particular focus will be paid to an emerging avenue of research; symmetry-preserving observer design. Recent theory developments in this area have allowed limitations in the global convergence properties of infinite-dimensional observers to be overcome.

\subsection{Infinite-Dimensional Observers}
In many real world systems the dependent variables are functions of one or more spatial variables. An example would be the dynamics of waves in a body of water. The height of the surface varies continuously along the $x$ and $y$ directions. These spatial variables vary continuously, meaning an infinite number of parameters is required to describe the state of the system. Such systems are termed \textit{infinite-dimensional systems}, or \textit{distributed parameter systems}. Their dynamics are modelled by a partial differential equation (PDE). 

When a state estimate is required but direct measurement of the state with sensors is difficult or impossible, a \textit{state observer} is employed. A state observer is a filter that provides an estimate of the state of a system using the difference between its measured and predicted outputs. A more detailed description of an observer is provided in section \ref{sec:observerequations}. An observer for an infinite-dimensional system is called an \textit{infinite-dimensional observer}.

\subsubsection{Linear Systems}
Observer theory for \textit{linear} infinite-dimensional systems has been widely studied. The techniques used are typically extensions of Luenberger observers and Kalman filter methods used to observe finite dimensional systems.

A simplified approach is to use a spatial discretisation method such as finite difference or finite element to reduce the infinite-dimensional system to a finite-dimensional one. From here, finite-dimensional observer design techniques can be used. This is known as the \textit{early lumping} method, and was employed by Stavroulakis \cite{stavroulakis1973design} who implemented a finite-dimensional observer as part of a control system for an infinite dimensional systems.

The early lumping approach suffers from \textit{spillover}, a phenomenon where performance is affected by the neglected dynamics of the system\cite{meirovitch1983problem}. Harkort \cite{harkort2011finite} recently developed an observer based control scheme that reduced this effect by using modelled outputs rather than true measurements to reduce the effect of the neglected dynamics.

More accurate observers can be designed with the \textit{late lumping} approach which uses the infinite-dimensional model of the system in the observer design. The result is an infinite-dimensional observer that is discretised later for practical implementation. These methods are typically extensions of Kalman or Luenberger methods to infinite dimensions. 

Early work by Gressang \cite{gressang1975observers} extended the Luenberger observer to infinite-dimensional systems whose state space was an abstract Banach space whose dynamics were defined by an infinitesimal generator of a semigroup. More recently, Smyshlyaev \cite{smyshlyaev2005backstepping} developed an exponentially converging backstepping observer for systems governed by parabolic PDEs. Ramdani introduced forward and backward observers \cite{ramdani2010recovering} whose convergence properties were investigated by Haine \cite{haine2014recovering}.

\subsubsection{Nonlinear Systems}
There is currently no universal approach for observer design for nonlinear infinite-dimensional systems.
The most common approach has been to linearise the system, then apply a linear infinite-dimensional observer design. Common linearisation methods are Lyapunov methods, extended linearisation and the Lie-algebraic approach \cite{primbs1996survey}.

There has been some progress in infinite-dimensional observer design for special cases of nonlinear systems. For bilinear systems, Xu \cite{xu1995observer} designed an infinite-dimensional observer that converged for certain inputs. Bounit \cite{bounit1997observers} designed Kalman and Luenberger type observers for infinite-dimensional bilinear systems. 

Despite these small advances in special case nonlinear design, the most common design methods for nonlinear infinite-dimensional systems rely on linearisation techniques. These techniques rely on the fact that differentiable functions can be approximated by a first-order Taylor expansion around a point. Luenberger and Kalman methods can be applied to linear approximations of system infinite-dimensional systems around an equilibrium point. This simplification relies on the dynamics of system at the point of linearisation being representative of the entire space. In general, this is not necessarily true, and is the biggest limitation in this design technique. The result is that these linearised observers only converge if the initial state estimate is within a local neighbourhood of the true state. Global converge is not guaranteed which severely limits robustness.

Global convergence can be achieved by taking account the symmetries inherent to the system during observer design. A powerful tool for dealing with symmetries is the theory of \textit{Lie groups}. Investigation into \textit{symmetry-preserving} observer design for systems on Lie groups is an active area of research, promising to produce theoretically validated design principles for nonlinear infinite-dimensional observers.

\subsection{Symmetry-Preserving Observers}
The motivation behind symmetry-preserving observers is to take advantage of invariances in the dynamics of the system. The goal is to design an observer around an equilibrium point in such a way that it can be extended converge around a wider set of points.

\subsubsection{Early work}
Geometry conscious observer design is not a new idea. Early investigation by Marcus \cite{marcus1984algebraic} into algebraic and geometric methods for nonlinear filter design showed promise.
A seminal work by Salcudean \cite{salcudean1991globally} was the design of an eventually exponential, globally converging observer for the attitude of rigid bodies. This design HOW DID IT WORK?
Another important result that is a precursor to the active research of today is by Aghannon \& Rouchon \cite{aghannan2002invariant} - invariant observer construction based on Cartan's moving frame method. Proved convergence for specific problem, but general case still open problem.
Usefulness show by Maithripala \cite{maithripala2005intrinsic} who used Aghannon's method in an intrinsic observer based controller - meaning performance independent of coordinates of configuration space.

\subsubsection{Active research}
Currently, there are two active research groups. Both have begun to apply symmetry methods to infinite dimensional observers too.

First group, Bonnabel, Auroux, Rouchon, Martin
Design observer around equilibrium point - extend to nonlinear - behaves well around continuum of equilibrium points. 
invariant observer - luenberger observer where invariant frame used to construct invariant output error.
2005 \cite{bonnabel2005invariant} Design procedure based on Aghannon's work. Achieved asymptotic stability - though this was done with design procedure not general, tailored to specific nonlinearities of the system.
2008 \cite{bonnabel2008symmetry} invariant error equation simplified convergence analysis. Global behaviour better, region of attraction larger in comparison to naively linearised observers.
2009 \cite{bonnabel2009non} For particular class of invariant systems, converge locally around any trajectory, global behaviour independent of trajectory.
2011 \cite{auroux2011symmetry} Infinite dimensional system - fluid in a water tank where height varies with position, time - continous variables - faster, more robust than previous attempts at infinite dimensional observer design with EKF.

Second group: ANU, Trumpf, Mahony\\
including Trumpf, Mahony, Hamel, Lageman
2008 \cite{mahony2009nonlinear} - nonlinear filters - special orthogonal group - attitude estimation - 3 observers with almost globally stable observer error
2012 \cite{trumpf2012analysis} - attitude observer - almost globally asymptotic, locally exponential
2013 \cite{mahony2013observers} - design methodology for 2 classes of systems- lift kinematic system onto symmetry group. design observer for the lifted system. - lyapunov method for innovation term. Basically, less general than Bonnabel group, stronger global convergence characteristics

Additional work:
Zarrouati \cite{zarrouati2013augmented} - PhD thesis - observer from rotation invariant equations for light and depth - camera and depth sensor - dense sensors give infinite dimensional measurement, though finite dimensional state estimated

Most recent work by Adarve \cite{adarvefiltering} is similar exploration into dense sensing for estimation of infinite dimensional state. Will take a close look at this work as it is inspiration for this project, similar in spirit.

Adarve et al. design an update-propagation filter to iteratively compute dense optical flow from CCD camera measurements. Rather than computing the optical flow independently at each frame, a two-stage process is used to build it incrementally. The propagation stage uses a non-linear PDE to model the transport of the optical flow in the next time step. The update stage corrects this prediction using the current image.

This paper is relevant to the project because:
\begin{itemize}
%\setlength{\itemsep}{1pt}
\item The use of a dense sensor allows for the image to be treated as continuous in time and space;
\item The state that is being observed is infinite dimensional;
\item The relationship between the state to observe $ \Phi $ and the measurement of the environment $ Y $ is given by a transport PDE;
\end{itemize}

The iterative filter used in this approach is an observer that estimates the state of the continuous spatio-temporal flow field. By using a dense sensor, the measured image stream can be treated as a continuous, infinite dimensional state. This is in contrast to sparse optical flow computation where the image is modelled as a set of discrete pixel values. However, the flow field is computed in support regions around a discrete set of control points. Here, this approach differs from a general infinite dimensional invariant observer might look like. The state is treated as a discrete set of locally continuous states which does not allow for symmetry considerations. \textit{Why? - the pde relations in the local regions are invariant to 2D rotation + translation, but what about the interaction between regions? - still need to do more reading on how invariance can be used to improve observer design.}

$\Phi$ and $Y$ are discretized spatially at the beginning of the algorithm design. Discretizing later would allow for symmetry to be taken advantage of.

Also, in order to derive the transport identity in the state propagation stage it was assumed the $\Phi$ was locally continuous. This assumption is invalid in the case of multiple objects, and will lead to an incorrect solution at the boundaries of objects. This is dealt with by computing the partial derivatives in the Jacobian matrix with data from direction of the source of the flow. This is done by choosing either the forward or backward partial derivative kernel in the $x$ and $y$ directions. Lesson: sometimes assumptions that are not completely valid must be made for simplicities sake - the implementation can be designed to minimise the resulting errors.



