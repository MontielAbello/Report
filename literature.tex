\section{Literature Review}
\textbf{should this be a chapter???}

Use dense sensors to estimate infinite dimensional state of environment can be achieved through an infinite dimensional observer. Richer theory can be gained by taking advantage of symmetries of the system. Will review progress in infinite dimensional, symmetry preserving observers.

\subsection{infinite-dimensional observers}
In many real world systems the dependent variables are functions of one or more spatial variables. These spatial variables form a continuum of ??? which means an infinite number of parameters are required to describe the state. Such systems are termed infinite dimensional systems, or distributed parameter systems. Their dynamics are modelled by a partial differential equation (PDE). When a state estimate is required but direct measurement of the state with sensors is difficult or impossible, a state observer is employed. A state observer is a filter that provides an estimate of the state of a system using the difference between measured and predicted outputs of the system.

\subsubsection{linear}
Observer theory for infinite dimensional \textit{linear} systems has been widely studied. The techniques used are typically extensions of Luenberger observers and Kalman filter methods used to observe finite dimensional systems.

A simplified approach is to use a spatial discretisation method such as finite difference or finite element to reduce the infinite dimensional system to a finite dimensional one. From here, finite dimensional observer design techniques can be used. This is known as the early lumping method, and was employed by Stavroulakis \cite{stavroulakis1973design} who implemented a finite dimensional observer as part of a control system for an infinite dimensional systems.

The early lumping approach suffers from \textit{spillover}, a phenomenon where performance is affected by the neglected dynamics \cite{meirovitch1983problem}. Harkort \cite{harkort2011finite} recently developed an observer based control scheme that reduced this effect by using modelled outputs rather than true measurements to reduce the effect of the neglected dynamics.

More accurate observers can be designed with the late lumping approach which uses the infinite dimensional model of the system in the observer design itself. The result is what is an actual infinite dimensional observer that is discretised later for practical implementation. These methods are typically extensions of Kalman or Luenberger observers. 
Gressang \cite{gressang1975observers} extended the Luenberger observer to infinite dimensional systems whose state space was an abstract Banach space - dynamics defined by infinitesimal generator - sort of like basis ? of semigroup.

Smyshlyaev \cite{smyshlyaev2005backstepping} developed an exponentially converging backstepping observer for systems governed by parabolic PDEs.

Ramdani introduced forward and backward observers \cite{ramdani2010recovering}.

Haine \cite{haine2014recovering} applied Ramdani's method, studied convergence properties.

\subsubsection{nonlinear}
There is no universal approach for observer design for infinite-dimensional nonlinear systems\\
One approach is to linearise the system, then apply a method for infinite dimensional linear observer design. Common linearisation methods are Lyapunov methods, extended linearisation and the Lie-algebraic approach \cite{primbs1996survey}.
For the special case of bilinear systems, Xu \cite{xu1995observer} designed an infinite dimensional observer that converged given certain inputs. some methods for special case - infinite dimensional bilinear systems. Bounit \cite{bounit1997observers} designed Kalman and Luenberger type observers for infinite dimensional bilinear systems. 

Trend - many design methods for infinite dimensional nonlinear systems rely on linearisation techniques. Differentiable function can be approximated by first order taylor expansion around a point. Linear approximation of system around an equilibrium point. Theory of Kalman filters, Luenberger observers can be applied. This simplification relies on dynamics of system at the point -  behaviour not the same throughout. Result is that observer only converges if the initial state estimate is within a local neighbourhood of the true state. Global converge is not guaranteed. 
To achieve this, direction into geometry, symmetry of system. Particularly powerful when dealing with symmetries is theory of Lie groups.
Investigation into observers for systems on Lie groups:

\subsection{symmetry preserving observers}
The idea behind symmetry preserving observers is to take advantage of invariances of the system. Design an observer around an equilibrium point in such a way that it can be extended to a wider set of points.

\subsubsection{Early work}
Observer design taking geometry into account not a new idea. Early investigation by Marcus \cite{marcus1984algebraic} into algebraic and geometric methods for nonlinear filter design showed idea had promise.
An important work by Salcudean - eventually exponential, globally converging observer - rigid bodies, attitude.
Work that has led to research today is by Aghannon \& Rouchon \cite{aghannan2002invariant} - invariant observer construction based on Cartan's moving frame method. Proved convergence for specific problem, but general case still open problem.
Usefulness show by Maithripala \cite{maithripala2005intrinsic} who used Aghannon's method in an intrinsic observer based controller - meaning performance independent of coordinates of configuration space.


\subsubsection{Active research}
Currently, there are two active research groups. Both have begun to apply symmetry methods to infinite dimensional observers too.

First group, Bonnabel, Auroux, Rouchon, Martin
Design observer around equilibrium point - extend to nonlinear - behaves well around continuum of equilibrium points. 
invariant observer - luenberger observer where invariant frame used to construct invariant output error.
2005 \cite{bonnabel2005invariant} Design procedure based on Aghannon's work. Achieved asymptotic stability - though this was done with design procedure not general, tailored to specific nonlinearities of the system.
2008 \cite{bonnabel2008symmetry} invariant error equation simplified convergence analysis. Global behaviour better, region of attraction larger in comparison to naively linearised observers.
2009 \cite{bonnabel2009non} For particular class of invariant systems, converge locally around any trajectory, global behaviour independent of trajectory.
2011 \cite{auroux2011symmetry} Infinite dimensional system - fluid in a water tank where height varies with position, time - continous variables - faster, more robust than previous attempts at infinite dimensional observer design with EKF.

Second group: ANU, Trumpf, Mahony\\
including Trumpf, Mahony, Hamel, Lageman
2008 \cite{mahony2009nonlinear} - nonlinear filters - special orthogonal group - attitude estimation - 3 observers with almost globally stable observer error
2012 \cite{trumpf2012analysis} - attitude observer - almost globally asymptotic, locally exponential
2013 \cite{mahony2013observers} - design methodology for 2 classes of systems- lift kinematic system onto symmetry group. design observer for the lifted system. - lyapunov method for innovation term. Basically, less general than Bonnabel group, stronger global convergence characteristics

Additional work:
Zarrouati \cite{zarrouati2013augmented} - PhD thesis - observer from rotation invariant equations for light and depth - camera and depth sensor - dense sensors give infinite dimensional measurement, though finite dimensional state estimated

Most recent work by Adarve \cite{adarvefiltering} is similar exploration into dense sensing for estimation of infinite dimensional state. Will take a close look at this work as it is inspiration for this project, similar in spirit.

Adarve et al. design an update-propagation filter to iteratively compute dense optical flow from CCD camera measurements. Rather than computing the optical flow independently at each frame, a two-stage process is used to build it incrementally. The propagation stage uses a non-linear PDE to model the transport of the optical flow in the next time step. The update stage corrects this prediction using the current image.

This paper is relevant to the project because:
\begin{itemize}
%\setlength{\itemsep}{1pt}
\item The use of a dense sensor allows for the image to be treated as continuous in time and space;
\item The state that is being observed is infinite dimensional;
\item The relationship between the state to observe $ \Phi $ and the measurement of the environment $ Y $ is given by a transport PDE;
\end{itemize}

The iterative filter used in this approach is an observer that estimates the state of the continuous spatio-temporal flow field. By using a dense sensor, the measured image stream can be treated as a continuous, infinite dimensional state. This is in contrast to sparse optical flow computation where the image is modelled as a set of discrete pixel values. However, the flow field is computed in support regions around a discrete set of control points. Here, this approach differs from a general infinite dimensional invariant observer might look like. The state is treated as a discrete set of locally continuous states which does not allow for symmetry considerations. \textit{Why? - the pde relations in the local regions are invariant to 2D rotation + translation, but what about the interaction between regions? - still need to do more reading on how invariance can be used to improve observer design.}

$\Phi$ and $Y$ are discretized spatially at the beginning of the algorithm design. Discretizing later would allow for symmetry to be taken advantage of.

Also, in order to derive the transport identity in the state propagation stage it was assumed the $\Phi$ was locally continuous. This assumption is invalid in the case of multiple objects, and will lead to an incorrect solution at the boundaries of objects. This is dealt with by computing the partial derivatives in the Jacobian matrix with data from direction of the source of the flow. This is done by choosing either the forward or backward partial derivative kernel in the $x$ and $y$ directions. Lesson: sometimes assumptions that are not completely valid must be made for simplicities sake - the implementation can be designed to minimise the resulting errors.



