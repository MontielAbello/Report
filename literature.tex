\section{Literature Review}

\subsection{infinite-dimensional observers}
\textbf{linear:}\\
-observer theory for linear infinite-dimensional systems widely studied\\
-techniques used typically extensions of luenberger observers used for finite-dimensional systems.\\
-simplified approach: use spatial discretisation such as finite different/finite element to reduce infinite-dimensional to finite-dimensional observer. \cite{harkort2011finite,meirovitch1983problem}
-better to design infinite-dimensional observer and only discretise for numerical implementation. \cite{haine2014recovering,helton1976systems,ramdani2010recovering} \textbf{TODO:} \textit{describe these design methods.}\\
\textbf{nonlinear:}\\ 
-no universal approach for observer design for infinite-dimensional nonlinear systems\\
-some methods for special case - infinite dimensional bilinear systems. \cite{xu1995observer,bounit1997observers}
-for finite-dimensional nonlinear systems, common design methods are: linearisation (ie EKF), lyapunov method, sliding mode, high gain
\subsection{symmetry preserving observers}
\subsubsection{early work}
\subsubsection{bonnabel et al}
\subsubsection{trumpf, mahony et al}
\subsubsection{juan's work - in detail}
