\chapter{Experimental Data}
In order to validate the performance of the observer implementation, experimental data was collected with a Hokuyo UBG-04LX-F01 scanning laser range-finder.

Measurements were taken to:
\begin{itemize}
\item build a model of the noise characteristics of the Hokuyo UBG-04LX-F01 in order to more accurately simulate the performance of the observer;
\item observe the motion of a moving cube of known state to test the observer in real-world conditions.
\end{itemize}

Section \ref{sensor_noise} details how measurements were taken to develop the noise model.
Section \ref{testingdata} describes how experimental range measurements were taken and how the ground truth cube state was determined. This work is still ongoing as the data must be calibrated before it can be used to assess the performance of the observer. 

\section{Sensor Noise Characterisation} \label{sensor_noise}
An accurate range sensor simulation must include a model for the error distribution of the measurements. A noise model for the Hokuyo UBG-04LX-F01 used in this research was developed in by Park et al. \cite{park2010characterization}. The effect of range and incidence angles on the error was measured, but a unified model combining both was not provided. Furthermore, \cite{park2010characterization} showed that the measurement error depends highly on the texture and colour of the surface measured. To accurately model the Hokuyo UBG-04LX-F01 for the usage case of this research, a wide range of measurements using a specific surface were taken to determine the effect of range and incidence angle on the error distribution.

	\subsection{Measurement Setup}
	A flat surface was painted matte white. The surface was placed perpendicular to the ground and at a known distance and angle with respect to the range sensor.
	1200 samples of the measured distance to the surface were taken.
	
	For this research, the cube is likely to be placed within 1.5m from the sensor and at any orientation. The range error distribution for these conditions should be measured. The distance from the sensor to the measurement surface was thus varied in 50mm increments between 250mm and 1750mm, to an accuracy of $\pm$1mm. At each of these ranges the incidence angles was varied in 20$^\circ$ increments from 0$^\circ$ to 80$^\circ$ to an accuracy of $\pm$0.5$^\circ$. The physical setup is shown in Figure \ref{fig:noise_setup}.
		 
	\subsection{Results}
		The range error $r_{error} = r_{ground truth} - r_{measured}$ was computed. The distributions of this error for varying ranges and angles is shown in Figure \ref{fig:mean_hist}. The range error is approximately normally distributed.
		\begin{figure}
		\centering
		  \includegraphics[width=1\textwidth,trim = 0mm 0mm 0mm 0mm,clip]{./Figures/range_error_histograms.jpg}
		  \caption{Sensor noise function $f_{UBG}(r,\theta)$ approximately normally distributed}
		  \label{fig:mean_hist}
		\end{figure}
		
		The mean range error as function of $r$ and $\theta$ is shown in Figure \ref{fig:mean_range_error}. The standard deviation of the range errors as function of $r$ and $\theta$ is shown in Figure \ref{fig:stddev_range_error}.
		\begin{figure}
	  		\centering
	  		\subfigure[\label{fig:mean_range_error_outliers}]{
	  		\begin{minipage}[b]{0.45\columnwidth}
    			\includegraphics[width=1\textwidth,trim = 0mm 0mm 0mm 0mm,clip]{./Figures/noise_mean_range_error}\vspace*{0ex}
	  		\end{minipage}}
	  		\subfigure[\label{fig:mean_range_error_no_outliers}]{
	  		\begin{minipage}[b]{0.45\columnwidth}
    			\includegraphics[width=1\textwidth,trim = 0mm 0mm 0mm 0mm,clip]{./Figures/noise_mean_range_error_removed_outliers}\vspace*{0ex}
		 \end{minipage}}
	  		\caption{mean range error ($r_{error} = r_{ground truth} - r_{measured}$) vs $(r,\theta)$ showing (a) large error at high angles and range, (b) overall shape}
	  		\label{fig:mean_range_error}
		\end{figure}
		
		\begin{figure}
	  		\centering
	  		\subfigure[\label{fig:stddev_range_error_outliers}]{
	  		\begin{minipage}[b]{0.45\columnwidth}
    			\includegraphics[width=1\textwidth,trim = 0mm 0mm 0mm 0mm,clip]{./Figures/noise_stddev_range_error}\vspace*{0ex}
	  		\end{minipage}}
	  		\subfigure[\label{fig:stddev_range_error_no_outliers}]{
	  		\begin{minipage}[b]{0.45\columnwidth}
    			\includegraphics[width=1\textwidth,trim = 0mm 0mm 0mm 0mm,clip]{./Figures/noise_stddev_range_error_removed_outliers}\vspace*{0ex}
	 		 \end{minipage}}
	  		\caption{range error $\sigma$ vs $(r,\theta)$ showing (a) outliers/large std dev at high angles and range, (b) overall shape}
	  		\label{fig:stddev_range_error}
		\end{figure}
		
		Figures \ref{fig:mean_range_error_outliers} and \ref{fig:stddev_range_error_outliers} show that the mean error and error standard deviation increase significantly when $\theta > 75^\circ$ and $r > 0.8$m. This can be explained by considering what happens to the laser beam under these conditions. Though it has been idealised as a ray in the simulation, the laser has a nonzero beam width. Thus, as $\theta$ increases, one side of the beam will encounter the surface before the centre of the beam. A portion of the light will reach the sensor earlier, though the total amount of light will be reduced as the angle increases. This earlier reflected light will result in a shorter range measurement, but less reflected light will cause a longer range measurement. For angles greater than 75$^\circ$ and ranges greater than 0.8m, the reflected light is insufficient to allow a range measurement. The sensor returns the maximum possible range measurement of 4095mm. In modelling the noise, range measurements for $\theta > 75^\circ$ and $r > 0.8$m are discarded.
		
		These results are corroborated by \cite{park2010characterization} who reported difficulty in acquiring measurements for high angles and modelled the noise distribution as Gaussian.
		
		4th degree polynomial surfaces in equations \ref{eq:mean} and \ref{eq:std} were fitted to the adjusted set of data points using Matlab's curve fitting tool. The surfaces and the goodness of fit are shown in Figure \ref{fig:surface_range_error}. 
		\begin{figure}
	  		\centering
	  		\subfigure[\label{fig:surface_mean_range}]{
	  		\begin{minipage}[b]{0.45\columnwidth}
    			\includegraphics[width=1\textwidth,trim = 0mm 0mm 0mm 0mm,clip]{./Figures/surface_mean_range_error}\vspace*{0ex}
	  		\end{minipage}}
	  		\subfigure[\label{fig:surface_stddev_range}]{
	  		\begin{minipage}[b]{0.45\columnwidth}
    			\includegraphics[width=1\textwidth,trim = 0mm 0mm 0mm 0mm,clip]{./Figures/surface_stddev_range_error}\vspace*{0ex}
		 \end{minipage}}
	  		\caption{polynomials fitted to range error mean \& standard deviation data points to model noise. (a) SSE: 0.003234, R-square: 0.8447, Adjusted R-square: 0.8278, RMSE: 0.005027(b) SSE: 7.592e-06, R-square: 0.9196, Adjusted R-square: 0.9103, RMSE: 0.0002515}
	  		\label{fig:surface_range_error}
		\end{figure}
		
		\subsubsection{Gaussian noise model}
		\begin{equation}
			\tilde{r}(r,\theta) = 
				\begin{cases}
					\hfill
					f_s(r,\theta,\phi(k)) = r + \mathcal{N}(\mu,\sigma)
						\hfill & \text{$\theta \leq 75^\circ$ or $r \leq 0.8$}\\
					NaN \hfill & \text{$\theta > 75^\circ$ and $r > 0.8$}
				\end{cases} 
		\end{equation}	
		where
		\begin{equation} \label{eq:mean}
			\begin{aligned}
				\mu = & a_{00} + a_{10}r + a_{01}\theta + a_{20}r^2 + a_{11}r\theta + a_{02}\theta^2\\
				      & + a_{30}r^3 + a_{21}r^2\theta + a_{12}r\theta^2 + a_{03}\theta^3 + a_{40}r^4 \\ 
				      & + a_{31}r^3\theta + a_{22}r^2\theta^2 + a_{13}r\theta^3 + a_{04}\theta^4
			\end{aligned}		
		\end{equation}
		\begin{equation} \label{eq:std}
			\begin{aligned}
				\sigma = & b_{00} + b_{10}r + b_{01}\theta + b_{20}r^2 + b_{11}r\theta + b_{02}\theta^2\\
			         	 & + b_{30}r^3 + b_{21}r^2\theta + b_{12}r\theta^2 + b_{03}\theta^3 + a_{40}r^4 \\ 
			         	 & + b_{31}r^3\theta + b_{22}r^2\theta^2 + b_{13}r\theta^3 + b_{04}\theta^4
			\end{aligned}
		\end{equation}
		
		and coefficients $a_{ij}$ and $b_{ij}$ are provided in tables \ref{tab:noise_a} and \ref{tab:noise_b} respectively.
				\begin{table}[h!]
  \centering
  \caption{$a_{ij}$ coefficients}
  \label{tab:noise_a}
  \begin{tabular}{c| c c c c c}
     	  & $j_0$ 	 & $j_1$   & $j_2$ 	 & $j_3$   & $j_4$ \\
    \hline
   	$i_0$ & -0.06529 & 0.2126  & -0.533	 & 0.4629  & -0.1223 \\
   	$i_1$ & 0.2024   & -0.1906 & 0.4006	 & -0.1791 & 0 \\
   	$i_2$ & -0.3074  & 0.0228  & -0.0716 & 0 	   & 0 \\
   	$i_3$ & 0.2053   & 0.01455 & 0 		 & 0 	   & 0 \\
   	$i_4$ & -0.04912 & 0 	   & 0 		 & 0 	   & 0 \\
  \end{tabular}
\end{table}



				\begin{table}[h!]
  \centering
  \caption{$b_{ij}$ coefficients}
  \label{tab:noise_b}
  \begin{tabular}{c| c c c c c}
		  & $j_0$ 	  & $j_1$    & $j_2$ 	  & $j_3$	  & $j_4$ \\
	\hline
	$i_0$ & 0.001242  & 0.2126	 & -0.01128	  & 0.01162	  & -0.002746 \\
	$i_1$ & 0.00352   & 0.006146 & 0.01021	  & -0.007316 & 0 \\
	$i_2$ & -0.005138 & -0.00626 & -0.0005068 & 0 		  & 0 \\
	$i_3$ & 0.004067  & 0.001337 & 0 		  & 0 		  & 0 \\
	$i_4$ & -0.001092 & 0 		 & 0 		  & 0 		  & 0 \\
  \end{tabular}
\end{table}



	

		\subsubsection{Surface noise}
		An additional source of error was observed and found to be mostly independent of $r$ and $\theta$. This may be caused by surface properties of the environment, though the error is larger than expected in this case. A possible explanation is compensation performed by sensor to produce globally straight lines. While flat surfaces do appear flat from a distance, locally there are regular variations in depth as shown in Figure \ref{fig:measured_surface_noise}. 
		
		This surface noise was modelled with a random walk function
		\begin{equation}
			e_{surface} = a\sum_{n = 1}^{n_{Steps}}-1 + 2\:\left \lfloor{\mathcal{R}}\right \rfloor 
		\end{equation} 
		where $\mathcal{R}$ is a random variable following a uniform distribution on [0,1]. A step size $a = 0.0005m$ was used. Figure \ref{fig:surface_noise} shows that this model accurately models the measured surface variations. It should be noted that the measured variation appears concave while the simulated noise appears convex. This is due to the nature of the random walk noise. Over a large sample, both concave and convex surface noise is observed in real-world measurements and the simulate random walk simulation.

		\begin{figure}
	  		\centering
	  		\subfigure[\label{fig:measured_surface_noise}]{
	  		\begin{minipage}[b]{0.45\columnwidth}
    			\includegraphics[width=1\textwidth,trim = 0mm 0mm 0mm 0mm,clip]{./Figures/measured_surface_noise}\vspace*{0ex}
	  		\end{minipage}}
	  		\subfigure[\label{fig:simulated_surface_noise}]{
	  		\begin{minipage}[b]{0.45\columnwidth}
    			\includegraphics[width=1\textwidth,trim = 0mm 0mm 0mm 0mm,clip]{./Figures/simulated_surface_noise}\vspace*{0ex}
 			\end{minipage}}
	  		\caption{Comparision of (a) measured and (b) simulated surface noise}
	  		\label{fig:surface_noise}
		\end{figure}

\section{Testing Data Collection} \label{testingdata}
	\subsection{Setup}
		physical setup - \ref{experimental_data} PICS IN APPENDIX\\
		Real world, less than ideal conditions.
		Sensor:\\
		Hokuyo UBG-04LX-F01 - 2D scanning laser range-finder. Measures range\\
		
		Robot arm: \\
		Kinova Jaco - can program to move with and record pose
		
		Target object: \\
		$0.1 \times 0.1$ m MDF cube, spray painted matte white
		
		configurations/motions: \\
			-stationary\\
			-rotating\\
			-translating\\					
		arm forward kinematics $\rightarrow$ cube pose\\
		estimate sensor angle with horizontal with wall calibration data		
		
	\subsection{Results}
		still need to calibrate data.
		However, prediction is observer wont work. First, neither range or continuity assumptions will hold - gripper hold cube will mess with things. Lots of noise.
		Likely will need an infinite dimensional observer, estimate entire depth field. Need symmetry to get robust convergence.
		Still, data is available for future research.

