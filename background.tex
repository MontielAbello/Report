\section{Theoretical Background}

\subsection{Rigid Body Dynamics}
	\subsubsection{Lie Groups: $\textbf{SE}(3)$}		
		\textbf{Lie groups}\\
		A Lie group $G$ is a group that also happens to be a differentiable manifold.
		
		Lie algebra
		
		matrix lie groups
		
		exponential map - Lie group generators
		
		actions
		
		adjoint map \& adjoint representation
		
		\textbf{3D space - $\mathbb{R}^3$}\\
		In practice, robot, sensor, environment exist in 3D Euclidean space - $\mathbb{R}^3$.\\
		An arbitrary function that maps a pose *(or point?) in $\mathbb{R}^3$ to another can be defined as:
		\begin{equation}
		f: \mathbb{R}^3 \rightarrow \mathbb{R}^3
		\end{equation}
		To represent rigid bodies, require mappings corresponding to rotation and translation. Translation can be modelled as a vector space ie $\mathbb{R}^3$ but the set of all rotations in $\mathbb{R}^3$ forms a Lie group.
		
		\textbf{Rotations}\\
		A rotation represents the motion of a rigid body about a fixed point. In $\mathbb{R}^3$ this is an isometry (a transformation that preserves distances between any pair of points) that has a determinant of +1 (proper isometries). The set of all proper orthogonal transformations is known as the \textit{special orthogonal group} $\textbf{SO}(3)$.
		
		A rotation about a point in $\mathbb{R}^3$ can be represented by: \textbf{TODO:} \textit{go into more detail on below}
		
		rotation matrix:\\
		$3\times3$ matrix where magnitude of each column is 1, columns are orthogonal, determinant is +1.
		
		scaled axis angle:\\
		3-vector where direction represents axis of rotation and magnitude represents angle of rotation.		
		
		quaternions:\\
		4-vector, same information as axis angle, but different form.\\
		
		\textbf{Translations}\\
		A one-to-one transformation that shifts each point by a constant vector ie. shifts origin *(haven't discussed coordinate frames yet) 

		\textbf{homogeneous representation}\\
		$4\times4$ screw matrix - represent rotation and translation with a single matrix of form:\\
		\begin{equation}
				\begin{bmatrix}
				  R		&	t \\
				  0		& 	1 
				\end{bmatrix}
		\end{equation}
		*show that 0 is 1x3\\
		OR\\
		*how to do nested matrices?
		\begin{equation}
		\begin{bmatrix}
		  0		&	0 	& 	0 	& 	0\\
		  0		& 	0  	& 	0 	& 	0\\
		  0		& 	0  	& 	0 	& 	0\\
		  0		& 	0 	& 	0 	& 	0
		\end{bmatrix}
		\end{equation}
		
		To apply a rigid transformation to a point $p = (x,y,z) $ in $\mathbb{R}^3$, represent with homogeneous coordinates. ie
		\begin{equation}
		p = 
		\begin{bmatrix}
				  x	\\
				  y	\\
				  z	\\
				  1	
		\end{bmatrix}
		\end{equation}
	
		\textbf{ $\textbf{SE}(3)$}\\
		
		
	\subsubsection{Reference Frames}
		-mathematical definitions\\
		-notation, diagrams
	\subsubsection{robot state representation}	
		-screw, twist, wrench matrices
	\subsubsection{robot dynamic model}
		-ODEs\\
		-update methods (euler, runge-kutta etc)
	\subsubsection{sensor dynamic model}
		-ODEs (same as robot + scanning)\\
		-parameters ie FOV, steps
\subsection{Symmetry Preserving Observers}
	\subsubsection{definitions?}
	\subsubsection{construction, ie moving frame method etc}
\subsection{Infinite Dimensional Observers}
\subsection{Discretisation Methods?}