\section{Theoretical Background}

\subsection{Rigid Body Dynamics}
	In practice, robot, sensor, environment exist in 3D Euclidean space - $\mathbb{R}^3$.
	To model the pose of a rigid body... (something about why Lie groups are necessary to represent pose)
			
	\subsubsection{Lie Groups}		
		A Lie group $\mathbf{G}$ is a group that is also a differentiable manifold.
		As a group, $\mathbf{G}$ is a set of elements and a group operation (denoted by multiplication, i.e. $AB$ for $A,B \in \mathbf{G}$) that satisfies the 4 group axioms:
		
		\begin{itemize}
		\item \textbf{Closure:} 
			The group operation
			$\mathbf{G} \times \mathbf{G} \mapsto \mathbf{G}$ 
			is a function that maps elements of $\mathbf{G}$ onto itself;
			$\forall A,B \in \mathbf{G}$, $AB \in \mathbf{G}$.
		\item \textbf{Associativity:} Elements of G are associative under the group operation;
			$\forall A,B,C \in \mathbf{G}$, $(AB)C=A(BC)$.
		\item \textbf{Identity:} There exists an identity element $I \in \mathbf{G}$  such that
			$\forall A \in \mathbf{G}$, $IA = AI = A$.
		\item \textbf{Inverse:} For all $A \in \mathbf{G}$ there exists an inverse element $A^{-1} \in \mathbf{G}$ such that $AA^{-1}=A^{-1}A=I$. 
		\end{itemize}
		
		Because the Lie group $\mathbf{G}$ is a differentiable manifold, it is locally Euclidean. This means that the neighbourhood around every element of $\mathbf{G}$ can be approximated with a tangent plane. This property allows calculus to be performed on elements of $\mathbf{G}$.
		
		\textbf{Matrix Lie groups}\\
			A matrix Lie group is made up of group elements which are $n \times n$ matrices.
			This work will be focus on matrix Lie groups because the exponential map and Lie bracket functions given below only apply to such Lie groups.
		
		\textbf{Lie algebra}\\
			The tangent space at the identity element of a Lie group is called the Lie algebra $\mathfrak{g}$. It is called the Lie \textit{algebra} because it has a binary operation, known as the Lie bracket $[X,Y]$. For matrix Lie groups the Lie bracket is
			\begin{equation}
				[A,B] \stackrel{\Delta}{=} AB-BA
			\end{equation}
			
		\textbf{The exponential map and logarithm map}\\		
			The mapping from the Lie algebra $\mathfrak{g}$ to the Lie group $\mathbf{G}$ is called the exponential map:
			\begin{equation}
				\exp: \mathfrak{g} \rightarrow \mathbf{G}
			\end{equation}			
			Similarly, the logarithm map maps elements from $\mathbf{G}$ to $\mathfrak{g}$:
			\begin{equation}
				\log: \mathbf{G} \rightarrow \mathfrak{g}
			\end{equation}
						
		\textbf{sub-heading?}\\
			For an \textit{n}-dimensional matrix Lie group, the Lie algebra $\mathfrak{g}$ is a vector space isomorphic to $\mathbb{R}^n$. The hat operator $\hat{\:}$ maps vectors $x \in \mathbb{R}^3$ to elements of $\mathfrak{g}$.				
			\begin{equation}
				\hat{\:}: x \in \mathbb{R}^n \rightarrow \hat{x} \in \mathfrak{g}
			\end{equation}		
			For a matrix Lie group $\mathbf{G}$ whose elements are $\textit{n} \times \textit{n}$ matrices, the elements of $\mathfrak{g}$ will also be $\textit{n} \times \textit{n}$ matrices. The hat operator is defined
			\begin{equation}
				\hat{x} = \sum\limits_{i=1}^n x_iG^i 
			\end{equation}
			where $G^i$ are $\textit{n} \times \textit{n}$ matrices known as the infinitesimal generators of $\mathbf{G}$.
						
		\textbf{Lie bracket and group operation}\\					
			For Lie groups endowned with the commutative property ($\forall A,B \mathbf{G}, AB = BA$), vector addition in the Lie algebra maps to a group operation in the Lie group. For $C = A + B$ where $A,B,C \in \mathfrak{g}$,
			\begin{equation}
				e^C = e^{A+B} = e^Ae^B
			\end{equation}
			For non-commutative Lie groups, the relationship between the Lie bracket and group operation does not hold. Instead, for $C = \log{e^Ae^B}$, $C$ is calculated with the Baker-Campbell-Hausdorff formula:
			\begin{equation}
				C = A + B + \frac{1}{2}[A,B] + \frac{1}{12}[A-B,[A,B]] + \frac{1}{24}[B,[A,[A,B]]] + \dots
			\end{equation}	
		
		\textbf{Actions}\\
			When a group action for a Lie group G acting on a manifold $M$ is a differentiable map, this is known as a Lie group action. For example, 3D rotations act on 3D points so the Lie group $\mathbf{SO}(3)$ acts on $\mathbb{R}^3$. A left action of $\mathbf{G}$ on $M$ is defined as a differentiable map
			\begin{equation}
				\Phi: \mathbf{G} \times M \mapsto M
			\end{equation}
			where
			\begin{itemize}
			\item the identity element $I$ maps M onto itself *(is that the right wording?)
				\begin{equation}
					\Phi(I,m) = m \textnormal{, } \forall m \in M
				\end{equation}
			\item Group actions compose according to
				\begin{equation}
					\Phi(m,\Phi(n,o)) = \Phi(mn,o)
				\end{equation}
			\end{itemize}
			
		\textbf{Adjoint map}\\		
		EXPLANATION - Sometimes before a function B can be applied on manifold acted on by a group action A, it is necessary to apply the conjugate of A to B????\\
		For $A \in \mathbf{G}$ define a function $\Psi$, known as the adjoint map of $\mathbf{G}$:
		\begin{equation}
			\Psi_A: \mathbf{G} \rightarrow \mathbf{G} \textnormal{, }
			\Psi_A(B) \stackrel{\Delta}{=} ABA^{-1}
		\end{equation}
		Taking the derivative:
		\begin{equation}
			\frac{\partial}{\partial t} \Psi_A(B(t))|_{t=0} = AVA^{-1} \textnormal{, }
			V \stackrel{\Delta}{=} 	\frac{\partial}{\partial t}B(t)|_{t=0}
		\end{equation}
		The adjoint	representation of $\mathbf{G}$ is given by the mapping
		\begin{equation}
			\textbf{Adj}_A: \mathfrak{g} \rightarrow \mathfrak{g} \textnormal{, }
			\textbf{Adj}_A(V) \stackrel{\Delta}{=} AVA^{-1}
		\end{equation}
	
		
	\subsubsection{\textbf{SO}(3)}	
		A rotation represents the motion of a point about the origin of a Euclidean space. In $\mathbb{R}^3$ this is a proper isometry: a transformation that preserves distances between any pair of points and has a determinant of +1. The set of all rotations about the origin of $\mathbb{R}^3$ is known as the \textit{special orthogonal group} $\textbf{SO}(3)$.
		This matrix Lie group is a subgroup of the general linear group $\textbf{GL}(3)$, so its group elements are $3 \times 3$ invertible matrices. These group elements are orthogonal matrices so their columns and rows are orthogonal unit vectors. 
		
		\textbf{Lie algebra}\\
		The Lie algebra $\mathfrak{so}(3)$ is vector space of $3 \times 3$ skew-symmetric matrices $\hat{\omega}$, where $\omega$ is a 3-vector representing an angular velocity. The direction of $\omega$ indicates the axis of rotation while its magnitude gives the angular velocity. 
		Elements of $\mathfrak{so}(3)$ are mapped to $\textbf{SO}(3)$ according to the exponential map:
		\begin{equation}
			\begin{split}
				\exp: \mathfrak{so}(3) \rightarrow \mathbf{SO}(3)\\
				[\omega]_\times \rightarrow \mathbf{R}_{3x3}
			\end{split}		
		\end{equation}		
		i.e. $\forall \omega \in \mathfrak{so}(3) \textnormal{, } \exp([\omega]_\times) \in  \mathbf{SO}(3)$
		
		Conversely, the logarithm map maps $3 \times 3$ rotation matrices of $\mathbf{SO}(3)$ to elements of $\mathfrak{so}(3)$:
		\begin{equation}
			\begin{split}
				\log: \mathbf{SO}(3) \rightarrow \mathfrak{so}(3)\\
				 \mathbf{R}_{3x3} \rightarrow [\omega]_\times
			\end{split}		
		\end{equation}		
		i.e. $\forall \mathbf{R} \in \mathbf{SO}(3)  \textnormal{, } \log(\mathbf{R}) \in  \mathfrak{so}(3)$
				
		\textbf{Actions}\\
		By the group action, elements of $\mathbf{SO}(3)$ rotate points in $\mathbb{R}^3$ about the origin. 
		\begin{equation}
			\Phi: \mathbf{SO}(3) \times \mathbf{R}^3 \mapsto \mathbf{R}^3
		\end{equation}
		
		\textbf{Adjoint map}
		\begin{equation}
			\Psi_R: \mathbf{SO}(3) \rightarrow \mathbf{SO}(3) \textnormal{, }
			\Psi_R(A) \stackrel{\Delta}{=} RAR^{-1}
		\end{equation}
		Taking the derivative:
		\begin{equation}
			\frac{\partial}{\partial t} \Psi_R(A(t))|_{t=0} = RBR^{-1} \textnormal{, }
			B \stackrel{\Delta}{=} 	\frac{\partial}{\partial t}A(t)|_{t=0}
		\end{equation}
		The adjoint	representation of $\mathbf{SO}(3)$ is given by the mapping
		\begin{equation}
			\textbf{Adj}_R: \mathfrak{so}(3) \rightarrow \mathfrak{so}(3) \textnormal{, }
			\textbf{Adj}_R(B) \stackrel{\Delta}{=} RBR^{-1}
		\end{equation}
		???? Hard to explain practical application without discussing reference frames: ie if position defined in body fixed frame but some other transformation defined in inertial frame. First undo rotation to get pose in inertial frame, apply transformation, then re-apply rotation
		
		TODO: more explanation needed here!
		
		\textbf{Rotation representation}\\		
		There are many conventions by which elements of $\mathbf{SO}(3)$ can be represented. Those  A rotation about a point in $\mathbb{R}^3$ can be represented by: \textbf{TODO:} \textit{go into more detail on below}
		
		\textbf{Rotation matrix}\\
		$3 \times 3$ matrix where magnitude of each column is 1, columns are orthogonal, determinant is +1. Group operation matrix is multiplication. Left action is left multiplication of point. 
		
		\textbf{Scaled axis}\\
		3-vector where direction represents axis of rotation and magnitude represents angle of rotation. Group operation - vector addition? Left action - rodrigues' rotation formula converts scaled axis to rotation matrix which is used to rotate point.
				
		\textbf{Quaternion}\\
		4-vector, same information as axis angle, but different form. Given axis of rotation $\mathbf{r}$ and angle of rotation $\theta$:
		\begin{equation}
			\mathbf{q} = 
			\begin{bmatrix}
				w \\
				x \\
				y \\
				z
			\end{bmatrix}
			 = 
			 \begin{bmatrix}
 				w \\
 				\mathbf{v}
			 \end{bmatrix}
			 =
			 \begin{bmatrix}
			 	\cos(\theta/2) \\
			 	\sin(\theta/2)\mathbf{r}
			 \end{bmatrix}
		\end{equation}
		Group operation - quaternion multiplication defined as:
		\begin{equation}
			\mathbf{q}_1\mathbf{q}_2 =
			\begin{bmatrix}
			 	w_1 \\
			 	\mathbf{v}_1
			\end{bmatrix} 
			\begin{bmatrix}
			 	w_2 \\
			 	\mathbf{v}_2
			\end{bmatrix} 
			=
			\begin{bmatrix}
			 	w_1w_2 - \mathbf{v}_1 \cdot \mathbf{v}_2 \\
			 	w_1\mathbf{v}_2 + w_2\mathbf{v}_1 + \mathbf{v}_1 \times \mathbf{v}_2
			\end{bmatrix} 
		\end{equation} 
		As with rotation matrices, quaternion multiplication is associative but not commutative.
		
	\subsubsection{\textbf{SE}(3)}	
		The special Euclidean group $\textbf{SE}(3)$ represents rigid transformation in $\mathbb{R}^3$. This is a matrix Lie group whose elements are (or should it be ``can be represented by''?) $4 \times 4$ matrices of the form
		\begin{equation}
			\textbf{X} = 
			\begin{bmatrix}
				  \mathbf{R}	&	\mathbf{t} \\
				  \textbf{0}_{1 \times 3}		& 	1 
			\end{bmatrix}
		\end{equation}
		where $\mathbf{R} \in \mathbf{SO}(3)$ and 
		$\mathbf{t} = 
		\begin{bmatrix}
			t_x	& t_y & t_z				
		\end{bmatrix}
		^\top \in \mathbb{R}^3$.
		
		$\textbf{SE}(3)$ is a semidirect product of $\textbf{SO}(3)$ and $ \mathbb{R}^3$. As its group elements contain a rotation matrix and translation vector, $\textbf{SE}(3)$ has 6 degrees of freedom and is a 6-dimensional manifold.
			
		\textbf{Lie algebra}\\
		The Lie algebra $\mathfrak{se}(3)$ is a vector space whose elements are $4 \times 4$ matrices of the form
		\begin{equation}
			\begin{bmatrix}
				  [\mathbf{\omega}]_\times	&  \mathbf{v}\\
				  \textbf{0}_{1 \times 3} & 0						  
			\end{bmatrix}
		\end{equation}
		where $\mathbf{\omega} =
		\begin{bmatrix}
			\omega_x & \omega_y & \omega_z				
		\end{bmatrix}
		^\top \in \mathbf{so}(3)$, representing an angular velocity in scaled axis representation, and
		$\mathbf{v} = 
		\begin{bmatrix}
			v_x & v_y & v_z				
		\end{bmatrix}
		^\top \in \mathbb{R}^3$, representing a linear velocity vector.
		
		Elements of $\mathfrak{se}(3)$ are mapped to $\textbf{SE}(3)$ according to the exponential map:
			\begin{equation}
				\begin{split}
					\exp: \mathfrak{se}(3) \rightarrow \mathbf{SE}(3)\\
					\begin{bmatrix}
						  [\mathbf{\omega}]_\times	&  \mathbf{v}\\
						  \textbf{0}_{1 \times 3} & 0						  
					\end{bmatrix}
					\rightarrow 
					\begin{bmatrix}
						  \mathbf{R}	&	\mathbf{t} \\
						  \textbf{0}_{1 \times 3}		& 	1 
					\end{bmatrix}
				\end{split}		
			\end{equation}		
			i.e. $\forall \mathbf{T} \in \mathfrak{se}(3) \textnormal{, } \exp(\mathbf{T}) \in  \mathbf{SE}(3)$
			
			Conversely, the logarithm map maps elements of $\mathbf{SE}(3)$ to elements of $\mathfrak{se}(3)$:
			\begin{equation}
				\begin{split}
					\log: \mathbf{SE}(3) \rightarrow \mathfrak{se}(3)\\
					\begin{bmatrix}
						\mathbf{R}	&	\mathbf{t} \\
						\textbf{0}_{1 \times 3}		& 	1 					 				  
					\end{bmatrix}
					\rightarrow 
					\begin{bmatrix}
					 	[\mathbf{\omega}]_\times	&  \mathbf{v}\\
					 	\textbf{0}_{1 \times 3} & 0			
					\end{bmatrix}
				\end{split}		
			\end{equation}		
			i.e. $\forall \mathbf{S} \in \mathbf{SE}(3)  \textnormal{, } \log(\mathbf{S}) \in  \mathfrak{se}(3)$
		
		\textbf{Actions}\\
		$\mathbf{SE}(3)$ group elements acts to perform a rigid transformation on points in $\mathbb{R}^3$. This corresponds to a rotation about the origin and a translation.
		To apply a transformation using the $4 \times 4$ matrix elements of $\mathbf{SE}(3)$ to a point $\textbf{p} = (x,y,z) $ in $\mathbb{R}^3$, the point must be represented with homogeneous coordinates: (is $p'$ okay for homogeneous points? $\hat{\:}$ is already used for skew-symmetric matrix)
		\begin{equation}
			\mathbf{p'} = 
			\begin{bmatrix}
				  \mathbf{p} \\
				  1	
			\end{bmatrix} =
			\begin{bmatrix}
				  x	\\
				  y	\\
				  z	\\
				  1	
			\end{bmatrix}
		\end{equation}
		The left group action of $\mathbf{SE}(3)$ is now simply a left matrix multiplication of $\mathbf{p}$:
		\begin{equation}
			\mathbf{p'}_1 = \mathbf{S}\mathbf{p'}_0 = 
			\begin{bmatrix}
				\mathbf{R}\mathbf{p}_0 + \mathbf{t}\\
				1	
			\end{bmatrix}
		\end{equation}
				
		\textbf{Adjoint Map}\\
		The adjoint map of $\mathbf{SE}(3)$ is
		\begin{equation}
			\Psi_S: \mathbf{SE}(3) \rightarrow \mathbf{SE}(3) \textnormal{, }
			\Psi_S(A) \stackrel{\Delta}{=} SAS^{-1}
		\end{equation}
		Taking the derivative:
		\begin{equation}
			\frac{\partial}{\partial t} \Psi_S(A(t))|_{t=0} = SBS^{-1} \textnormal{, }
			B \stackrel{\Delta}{=} 	\frac{\partial}{\partial t}A(t)|_{t=0}
		\end{equation}
		The adjoint	representation of $\mathbf{SE}(3)$ is given by the mapping
		\begin{equation}
			\textbf{Adj}_S: \mathfrak{se}(3) \rightarrow \mathfrak{se}(3) \textnormal{, }
			\textbf{Adj}_S(B) \stackrel{\Delta}{=} SBS^{-1}
		\end{equation}
		
		TODO: more explanation needed here!
		
	\subsubsection{Reference Frames}
		A reference frame is a system of coordinates that is used to uniquely identify points on a manifold. This report will deal with reference frames on $\mathbb{R}^3$, that are used both to define the position of a point and the pose of a rigid body in 3D space.
		Such a reference frame is represented by an element of \textbf{SE}(3).
		
		Consider three different reference frames, denoted \{A\},\{B\} and \{C\}.
		The notation $^{A}_{B}\mathbf{X}^{}_{C}$ defines the difference in $\mathbf{X}$ of the reference frame \{C\} with respect to the frame \{B\}, defined in the frame \{A\}.
		
		For example, $^{A}_{B}\mathbf{R}^{}_{C}$ defines the rotation of \{C\} with respect to \{B\}, defined in \{A\}.
		
		The notion of an inertial reference frame is introduced here. This will be defined as a reference frame that is stationary for the purpose of the problem being described. 
		
		\textbf{Pose:}\\
		The pose of a rigid body in a given reference frame is defined by its relative position and orientation with respect to the given reference frame and is represented by an element of \textbf{SE}(3). If a rigid body has orientation aligned with a reference frame \{C\} and position at the origin of \{C\}, then the pose of the rigid body with respect to \{B\} and defined in \{A\} is:
		\begin{equation}
			{^{A}_{B}\mathbf{S}^{}_{C}} = 
			\begin{bmatrix}
				^{A}_{B}\mathbf{R}^{}_{C}	& 	^{A}_{B}\mathbf{p}^{}_{C}\\
				\textbf{0}_{1 \times 3} & 1						  
			\end{bmatrix}
		\end{equation}		
		TODO: DIAGRAM HERE
		
		\textbf{Point:}\\
		To represent the position of a point in $\mathbb{R}^3$, the convention used will be to first define a reference frame with origin located as this point. The position of a point at the origin of \{C\} with respect to another point \{B\}, defined in terms of the reference frame \{A\} is a 4-vector in homogeneous coordinates:
		\begin{equation}
			{^{A}_{B}\mathbf{p'}^{}_{C}} = 
			\begin{bmatrix}
				{^{A}_{B}\mathbf{p}^{}_{C}} \\
				1
			\end{bmatrix} = 
			\begin{bmatrix}
				{^{A}_{B}x^{}_{C}} \\
				{^{A}_{B}y^{}_{C}} \\
				{^{A}_{B}z^{}_{C}} \\
				1
			\end{bmatrix}
		\end{equation}
		TODO: DIAGRAM HERE
		
		\textbf{Defining a point in terms of another reference frame:}\\
		Consider a point in $\mathbb{R}^3$ defined as the position of the frame \{B\} with respect to the frame \{A\}, defined in terms of the frame \{B\}. To redefine the point in terms of \{A\}, the left action of ${^{A}_{A}\mathbf{S}^{}_{B}} \in \mathbf{SE}(3)$ is used:
		\begin{equation}
			{^{A}_{A}\mathbf{p'}^{}_{B}} = {^{A}_{A}\mathbf{S}^{}_{B}}{^{B}_{A}\mathbf{p'}^{}_{B}}
		\end{equation}
		
		\textbf{Defining a pose in terms of another reference frame:}\\
		To define a pose transformation matrix in terms of a different reference frame, a matrix conjugation is used:
		\begin{equation}
			{^{B}_{C}\mathbf{X}^{}_{D}} = ({^{B}_{B}\mathbf{X}^{}_{A}}){^{A}_{C}\mathbf{X}^{}_{D}}({^{B}_{B}\mathbf{X}^{}_{A}})^{-1}
		\end{equation}

		\textbf{Inverse:}\\
		Taking the inverse of a pose transformation matrix has the effect of reversing the transformation, but does not alter the frame that the transformation is defined in terms of.
		\begin{equation}
			({^{A}_{B}\mathbf{X}^{}_{C}})^{-1} = {^{A}_{C}\mathbf{X}^{}_{B}}
		\end{equation}
	
	\subsubsection{Sensor State Representation}
		-inertial frame A, sensor/robot frame  B\\
		-p,v,a,R,omega,alpha - define, state reference frames\\	
		position $^{A}_{A}\mathbf{p}^{}_{B}$\\
		velocity $^{B}_{A}\mathbf{v}^{}_{B}$\\
		acceleration $^{B}_{A}\mathbf{a}^{}_{B}$\\
		orientation $^{A}_{A}\mathbf{R}^{}_{B}$ - rotation matrix\\
		angular velocity $^{B}_{A}\mathbf{\omega}^{}_{B}$ - scaled axis representation\\
		angular acceleration $^{B}_{A}\mathbf{\alpha}^{}_{B}$ - scaled axis representation
		
		pose of robot w.r.t. inertial frame, defined in inertial frame ie. screw matrix:
		\begin{equation}
				{^{A}_{A}\mathbf{S}^{}_{B}(t)} = 
				\begin{bmatrix}
						  ^{A}_{A}\mathbf{R}^{}_{B}(t) 	& 	^{A}_{A}\mathbf{p}^{}_{B}(t)\\
						  \textbf{0}_{1 \times 3} & 1						  
				\end{bmatrix}
		\end{equation}
		
		velocity (linear and angular) w.r.t. inertial frame, defined in body frame ie. twist matrix:
		\begin{equation}
				{^{B}_{A}\mathbf{T}^{}_{B}(t)} = 
				\begin{bmatrix}
		  {[^{B}_{A}\mathbf{\omega}^{}_{B}(t)]_\times} 	& 	^{B}_{A}\mathbf{v}^{}_{B}(t)\\
		  \textbf{0}_{1 \times 3} & 0						  
				\end{bmatrix}
		\end{equation}
				
		acceleration (linear and angular) w.r.t. inertial frame, defined in body frame ie. wrench matrix
		\begin{equation}
				{^{B}_{A}\mathbf{W}^{}_{B}(t)} = 
				\begin{bmatrix}
				  {[^{B}_{A}\mathbf{\alpha}^{}_{B}(t)]_\times} 	& 	^{B}_{A}\mathbf{a}^{}_{B}(t)\\
				  \textbf{0}_{1 \times 3} & 0						  
				\end{bmatrix}
		\end{equation}
		
		other parameters ie FOV, steps
				
	\subsubsection{Sensor Dynamic Model}
		screw matrix:
		\begin{equation}
			{\frac{\textnormal{d}}{\textnormal{d}t}} {^{A}_{A}\mathbf{S}^{}_{B}(t)} ={^{A}_{A}\mathbf{S}^{}_{B}(t)} {^{B}_{A}\mathbf{T}^{}_{B}(t)}
		\end{equation}		
		
		twist matrix:
		\begin{equation}
			{\frac{\textnormal{d}}{\textnormal{d}t}} {^{B}_{A}\mathbf{T}^{}_{B}(t)} = {^{B}_{A}\mathbf{W}^{}_{B}(t)}
		\end{equation}		
		
		wrench matrix:
		\begin{equation}
			{\frac{\textnormal{d}}{\textnormal{d}t}} {^{B}_{A}\mathbf{W}^{}_{B}(t)}=0			
		\end{equation}
		
		-something for scanning dynamics
		
		-update methods (euler, runge-kutta etc)
		
	\subsubsection{Object State Representation}
		-frame fixed to object - $B$
		-same as sensor + size $s$
	\subsubsection{Object Dynamic Model}
		-ODEs (same as sensor)
		
\subsection{Symmetry Preserving Observers}
	\subsubsection{definitions?}
	\subsubsection{construction, ie moving frame method etc}
\subsection{Infinite Dimensional Observers}
\subsection{Discretisation Methods?}