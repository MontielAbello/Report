\section{Theoretical Background}

\subsection{Rigid Body Dynamics}
	\subsubsection{Lie Groups}		
		\textbf{Definition: Lie group}\\
		The theory of Lie groups is required in the modelling of rigid bodies.
		A Lie group $\mathbf{G}$ is a group that is also a differentiable manifold.
		As a group, G is a set of elements plus a group operation (denoted $AB$ for $A,B \in \mathbf{G}$) that satisfies the 4 group axioms: \\
		closure: $\mathbf{G} \times \mathbf{G} \mapsto \mathbf{G}$ 
		ie $\forall A,B \in \mathbf{G}$, $AB \in \mathbf{G}$\\
		associativity: $\forall A,B,C \in \mathbf{G}$, $(AB)C=A(BC)$\\
		identity: $\exists I \in \mathbf{G}$ such that $IA = AI = A$, $\forall A \in \mathbf{G}$ \\
		inverse: $\forall A \in \mathbf{G}, \exists A^{-1}$ such that $AA^{-1}=A^{-1}A=I$\\
		Because G is a differentiable manifold, in a neighbourhood of the identity it approximates euclidean space and allows ...?
		
		\textbf{Matrix Lie groups}
		This work will be limited to matrix Lie groups, whose members  are $n \times n$ matrices.
		
		\textbf{Lie algebra}\\
		Tangent space of Lie group with origin at identity is called Lie algebra $\mathfrak{g} $ of the group. It is called the Lie \textit{algebra} because it has a binary operation, known as the Lie bracket $[X,Y]$. For matrix Lie groups, $[A,B] \stackrel{\Delta}{=} AB-BA$. Relationship to group operation, commutative \& non-commutative, Baker-Campbell-Hausdorff formula...
		
		\textbf{Exponential map}\\		
		exponential map - Lie group generators
			
		\textbf{Adjoint map}\\			
		adjoint map \& adjoint representation
		
		\textbf{Actions}\\
		Group element acting on manifold.
		left action
		
		
	\subsubsection{\textbf{SO}(3)}
		\textbf{Group elements}\\
		Subgroup of GL(3), $3 \times 3$ invertible matrices.
		A rotation represents the motion of a rigid body about a fixed point. In $\mathbb{R}^3$ this is an isometry (a transformation that preserves distances between any pair of points) that has a determinant of +1 (proper isometries). The set of all proper orthogonal transformations is known as the \textit{special orthogonal group} $\textbf{SO}(3)$.
		
		\textbf{Lie algebra}\\
		Lie algebra $\mathfrak{so}(3)$ is vector space of $3 \times 3$ skew-symmetric matrices $\hat{\omega}$.
		
		\textbf{Actions}\\
		Group action rotates point in $\mathbb{R}^3$.
		
		\textbf{Adjoint map}
		
		\textbf{Rotation representation}\\		
		A rotation about a point in $\mathbb{R}^3$ can be represented by: \textbf{TODO:} \textit{go into more detail on below}
		
		rotation matrix:\\
		$3 \times 3$ matrix where magnitude of each column is 1, columns are orthogonal, determinant is +1.
		
		scaled axis angle:\\
		3-vector where direction represents axis of rotation and magnitude represents angle of rotation.		
		
		quaternions:\\
		4-vector, same information as axis angle, but different form.
		
	\subsubsection{\textbf{SE}(3)}		
		\textbf{3D space - $\mathbb{R}^3$}\\
		In practice, robot, sensor, environment exist in 3D Euclidean space - $\mathbb{R}^3$.\\
		An arbitrary function that maps a pose *(or point?) in $\mathbb{R}^3$ to another can be defined as:
		\begin{equation}
		f: \mathbb{R}^3 \rightarrow \mathbb{R}^3
		\end{equation}
		To represent rigid bodies, require mappings corresponding to rotation and translation. Translation can be modelled as a function on a vector space $\mathbb{R}^3$ but the set of all rotations in $\mathbb{R}^3$ forms a Lie group. 

		\textbf{homogeneous representation}\\
		$4 \times 4$ screw matrix - represent rotation and translation with a single matrix of form:
		
		\begin{equation}
				\begin{bmatrix}
				  \mathbf{R}	&	\mathbf{t} \\
				  \textbf{0}_{1 \times 3}		& 	1 
				\end{bmatrix}
		\end{equation}
		\textbf{TODO:} align the R and 0!!!!

		To apply a rigid transformation to a point $\textbf{p} = (x,y,z) $ in $\mathbb{R}^3$, represent with homogeneous coordinates. ie
		\begin{equation}
		\mathbf{p} = 
		\begin{bmatrix}
				  x	\\
				  y	\\
				  z	\\
				  1	
		\end{bmatrix}
		\end{equation}
	
		\textbf{Elements of $\textbf{SE}(3)$}
		\begin{equation}
			\begin{bmatrix}
				\mathbf{R}	&	\mathbf{t} \\
			  	\textbf{0}_{1 \times 3}		& 	1 
			\end{bmatrix}
		\end{equation}
		
		\textbf{Lie algebra}
		\begin{equation}
			\begin{bmatrix}
				  [\mathbf{\omega}]_\times	&  \mathbf{v}\\
				  \textbf{0}_{1 \times 3} & 0						  
			\end{bmatrix}
		\end{equation}
		
		\textbf{Actions}\\
		group element acts on $\mathbb{R}^3$ - point in homogeneous coordinates
		
		\textbf{Adjoint Map}\\
		-adjoint map \& adjoint representation
		
	\subsubsection{Reference Frames}
		A reference frame is a system used to define a point on a manifold, on this case the Euclidean space $\mathbb{R}^3$. A reference frame is represented by an element of \textbf{SE}(3).\\
		$^{A}_{B}\mathbf{X}^{}_{C}$ defines transformation of C w.r.t. B defined in A
		
		Definition: Pose
			
		Definition: point
		-homogeneous coordinates		
		
		Inverse
		\begin{equation}
			({^{A}_{B}\mathbf{X}^{}_{C}})^{-1} = {^{A}_{C}\mathbf{X}^{}_{B}}
		\end{equation}
		
		Transform point from one reference frame to another:
		\begin{equation}
			{^{A}_{A}\mathbf{p}^{}_{B}} = {^{A}_{A}\mathbf{X}^{}_{B}}{^{B}_{A}\mathbf{p}^{}_{B}}
		\end{equation}
		\begin{equation}
			{^{B}_{A}\mathbf{p}^{}_{B}} = {^{B}_{B}\mathbf{X}^{}_{A}}{^{A}_{A}\mathbf{p}^{}_{B}}
		\end{equation}
		Transform pose from one reference frame to another	
		-change of basis
		\begin{equation}
			{^{B}_{C}\mathbf{X}^{}_{D}} = ({^{B}_{B}\mathbf{X}^{}_{A}}){^{A}_{C}\mathbf{X}^{}_{D}}({^{B}_{B}\mathbf{X}^{}_{A}})^{-1}
		\end{equation}

	
	\subsubsection{Sensor State Representation}
		-inertial frame A, sensor/robot frame  B\\
		-p,v,a,R,omega,alpha - define, state reference frames\\	
		position $^{A}_{A}\mathbf{p}^{}_{B}$\\
		velocity $^{B}_{A}\mathbf{v}^{}_{B}$\\
		acceleration $^{B}_{A}\mathbf{a}^{}_{B}$\\
		orientation $^{A}_{A}\mathbf{R}^{}_{B}$ - rotation matrix\\
		angular velocity $^{B}_{A}\mathbf{\omega}^{}_{B}$ - scaled axis representation\\
		angular acceleration $^{B}_{A}\mathbf{\alpha}^{}_{B}$ - scaled axis representation
		
		pose of robot w.r.t. inertial frame, defined in inertial frame ie. screw matrix:
		\begin{equation}
				{^{A}_{A}\mathbf{S}^{}_{B}(t)} = 
				\begin{bmatrix}
						  ^{A}_{A}\mathbf{R}^{}_{B}(t) 	& 	^{A}_{A}\mathbf{p}^{}_{B}(t)\\
						  \textbf{0}_{1 \times 3} & 1						  
				\end{bmatrix}
		\end{equation}
		
		velocity (linear and angular) w.r.t. inertial frame, defined in body frame ie. twist matrix:
		\begin{equation}
				{^{B}_{A}\mathbf{T}^{}_{B}(t)} = 
				\begin{bmatrix}
		  {[^{B}_{A}\mathbf{\omega}^{}_{B}(t)]_\times} 	& 	^{B}_{A}\mathbf{v}^{}_{B}(t)\\
		  \textbf{0}_{1 \times 3} & 0						  
				\end{bmatrix}
		\end{equation}
				
		acceleration (linear and angular) w.r.t. inertial frame, defined in body frame ie. wrench matrix
		\begin{equation}
				{^{B}_{A}\mathbf{W}^{}_{B}(t)} = 
				\begin{bmatrix}
				  {[^{B}_{A}\mathbf{\alpha}^{}_{B}(t)]_\times} 	& 	^{B}_{A}\mathbf{a}^{}_{B}(t)\\
				  \textbf{0}_{1 \times 3} & 0						  
				\end{bmatrix}
		\end{equation}
		
		other parameters ie FOV, steps
				
	\subsubsection{Sensor Dynamic Model}
		screw matrix:
		\begin{equation}
			{\frac{\textnormal{d}}{\textnormal{d}t}} {^{A}_{A}\mathbf{S}^{}_{B}(t)} ={^{A}_{A}\mathbf{S}^{}_{B}(t)} {^{B}_{A}\mathbf{T}^{}_{B}(t)}
		\end{equation}		
		
		twist matrix:
		\begin{equation}
			{\frac{\textnormal{d}}{\textnormal{d}t}} {^{B}_{A}\mathbf{T}^{}_{B}(t)} = {^{B}_{A}\mathbf{W}^{}_{B}(t)}
		\end{equation}		
		
		wrench matrix:
		\begin{equation}
			{\frac{\textnormal{d}}{\textnormal{d}t}} {^{B}_{A}\mathbf{W}^{}_{B}(t)}=0			
		\end{equation}
		
		-something for scanning dynamics
		
		-update methods (euler, runge-kutta etc)
		
	\subsubsection{Object State Representation}
		-frame fixed to object - $B$
		-same as sensor + size $s$
	\subsubsection{Object Dynamic Model}
		-ODEs (same as sensor)
		
\subsection{Symmetry Preserving Observers}
	\subsubsection{definitions?}
	\subsubsection{construction, ie moving frame method etc}
\subsection{Infinite Dimensional Observers}
\subsection{Discretisation Methods?}